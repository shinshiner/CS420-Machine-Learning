\documentclass[12pt,a4paper]{article}
\usepackage{amsmath,amscd,amsbsy,amssymb,latexsym,url,bm,amsthm}
\usepackage{epsfig,graphicx,subfigure}
\usepackage{enumitem,balance}
\usepackage{wrapfig}
\usepackage{mathrsfs, euscript}
\usepackage[usenames]{xcolor}
\usepackage{hyperref}
\usepackage[vlined,ruled,commentsnumbered,linesnumbered]{algorithm2e}
\usepackage{float}
\usepackage{array}
\usepackage{diagbox}
\usepackage{color}
\usepackage{indentfirst}
\usepackage{fancyhdr}
\usepackage{gensymb}
\usepackage{geometry}
\usepackage{setspace}
\usepackage{aurical}
\usepackage{times}
\usepackage{caption}
\usepackage{fontspec}
\usepackage{booktabs}
\setmainfont{Times New Roman}

\newtheorem{theorem}{Theorem}[section]
\newtheorem{lemma}[theorem]{Lemma}
\newtheorem{proposition}[theorem]{Proposition}
\newtheorem{corollary}[theorem]{Corollary}
\newtheorem{exercise}{Exercise}[section]
\newtheorem*{solution}{Solution}
\theoremstyle{definition}

\newcommand{\postscript}[2]
 {\setlength{\epsfxsize}{#2\hsize}
  \centerline{\epsfbox{#1}}}
\renewcommand{\baselinestretch}{1.0}

\setlength{\oddsidemargin}{-0.365in}
\setlength{\evensidemargin}{-0.365in}
\setlength{\topmargin}{-0.3in}
\setlength{\headheight}{0in}
\setlength{\headsep}{0in}
\setlength{\textheight}{10.1in}
\setlength{\textwidth}{7in}
\makeatletter \renewenvironment{proof}[1][Proof] {\par\pushQED{\qed}\normalfont\topsep6\p@\@plus6\p@\relax\trivlist\item[\hskip\labelsep\bfseries#1\@addpunct{.}]\ignorespaces}{\popQED\endtrivlist\@endpefalse} \makeatother
\makeatletter
\renewenvironment{solution}[1][Solution] {\par\pushQED{\qed}\normalfont\topsep6\p@\@plus6\p@\relax\trivlist\item[\hskip\labelsep\bfseries#1\@addpunct{.}]\ignorespaces}{\popQED\endtrivlist\@endpefalse} \makeatother

\begin{document}
\noindent
%==========================================================
\noindent\framebox[\linewidth]{\shortstack[c]{
\Large{\textbf{Report on Homework 2}}\vspace{1mm}\\ 
CS420, Machine Learning, Shikui Tu, Summer 2018 \vspace{1mm} \\
Zelin Ye 515030910468}}

\section{PCA algorithm}

\begin{algorithm}[H]
	\SetKwInOut{Input}{Input}
	\SetKwInOut{Output}{Output}
	\caption{A variant of k-means}
	\label{alg:kmean_vs_GMM}
	\vspace{0.25\baselineskip}
	
	\Input{The number of clusters $K$}
	\Output{$\pi_{k}, \mu_{k}, \Sigma_{k}, (k=1,2,...,K)$}
	Initialize the means $\mu_{k}$, covariances $\Sigma_{k}$, mixing coefficients $\pi_{k}$ and threshold $Thres$;
	
	Evaluating the initial value of the log likelihood;
	
	\While{the convergence criterion of parameters or log likelihood is not satisfied}{
	
		\textbf{E step.} Evaluate the responsibilities with the current parameter values:
		\BlankLine
		
		\begin{center}
			$\omega \leftarrow \dfrac{\pi_{k}\mathcal{N}(x_{n}|\mu_{k}, \Sigma_{k})}{\sum\limits_{j=1}^{K}\pi_{j}\mathcal{N}(x_{n}|\mu_{j}, \Sigma_{j})}$\quad $,$ \quad
			$\gamma(z_{nk}) \leftarrow \left\{
				\begin{aligned}
					\omega & & {\omega > Thres} \\
					0 & & {\omega \leq Thres}
				\end{aligned}
			\right.$
		\end{center}
		
		\begin{center}
			$z_{n} \leftarrow \dfrac{e^{z_{n_{i}}}}{\sum\limits_{j=1}^{K}e^{z_{n_{j}}}}$
		\end{center}
		
		\textbf{M step.} Re-estimate the parameters with the current responsibilities:
		
		\begin{center}
			$N_{k} \leftarrow \sum\limits_{n=1}^{N}\gamma(z_{nk})$\quad $,$ \quad
			$\mu^{new}_{k} \leftarrow \dfrac{1}{N_{k}}\sum\limits_{n=1}^{N}\gamma(z_{nk})x_{n}$
		\end{center}
		
		\begin{center}
			$\Sigma^{new}_{k} \leftarrow \dfrac{1}{N_{k}}\sum\limits_{n=1}^{N}\gamma(z_{nk})(x_{n}-\mu^{new}_{k})(x_{n}-\mu^{new}_{k})^{T}$
		\end{center}
		
		\begin{center}
			$\pi^{new}_{k} \leftarrow \dfrac{N_{k}}{N}$
		\end{center}
		
		Evaluate the log likelihood:
		
		\begin{center}
			ln $p(X|\mu,\Sigma,\pi) \leftarrow \sum\limits_{k=1}^{K}$ ln$\left\{\sum\limits_{k=1}^{K}\pi_{k}\mathcal{N}(x_{n}|\mu_{k}, \Sigma_{k})\right\}.$
		\end{center}
	}
	\Return $\pi_{k}, \mu_{k}, \Sigma_{k}, (k=1,2,...,K)$;
\end{algorithm}

\section{Factor Analysis (FA)}

\vspace{-0.03\linewidth}
\begin{large}
\begin{align*}
	\centering	
	p(y|x) &= \dfrac{p(x|y)p(y)}{p(x)} \\
	&= \dfrac{G(x|Ay+\mu,\Sigma_{e})G(y|0,\Sigma_{y})}{p(x)} \\
	&= \dfrac{G(x|Ay+\mu,\Sigma_{e})G(y|0,\Sigma_{y})}{G(x|\mu+\mu_{e},AA^{T}\Sigma_{y}+\Sigma_{e})}
\end{align*}
\end{large}

where $\mu_{e}$ denotes the mean value of $e$, generally considered to be 0.

\section{Independent Component Analysis (ICA)}

ICA aims to decompose the source signal into independent parts. If the source signal is non-Gaussian, the decomposition is unique, or there would be a variety of such decomposition.

\section{Causal discovery algorithms}

pass

\section{Causal tree reconstruction}

pass

\end{document}